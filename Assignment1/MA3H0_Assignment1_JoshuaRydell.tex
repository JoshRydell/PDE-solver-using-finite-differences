\documentclass[a4paper,11pt]{article}


\title{MA3H0 Assignment 1}
\author{Joshua Rydell}
\usepackage{graphicx}
\usepackage{framed}
\usepackage{amsmath}
\usepackage{amssymb}
\usepackage{amsfonts}
\usepackage{cite}
\usepackage{mathrsfs}
\usepackage{array}
\usepackage[numbers]{natbib}
\bibliographystyle{plainnat}
\usepackage{cite}

\usepackage{amsthm} %  package used to make the theorem environments work.
% code below sets up new theorem environments
\theoremstyle{plain} % this sets the style for all new environments created using \newtheorem to have the "theorem" style, which as a bold title, italic text and vertical space above and below it. 
\newtheorem{thm}{Theorem}[section] 
\newtheorem{lem}[thm]{Lemma} 
\newtheorem{prop}[thm]{Proposition} 
\newtheorem*{cor}{Corollary} 
\newtheorem*{claim}{Claim} 

\theoremstyle{definition} % this sets the style for all new environments created using \newtheorem to have the "definition" style, which as a bold title, upright text and vertical space above and below it.
\newtheorem{defn}{Definition}[section]
\newtheorem{eg}{Example}[section]


\theoremstyle{remark} % this sets the style for all new environments created using \newtheorem to have the "remark" style, which as an italic non-bold title, upright text and no extra vertical space above and below it.
\newtheorem*{rem}{Remark} 
\newtheorem{case}{Case}

% note that there is already an in-built "proof" environment which we do not need to create


\newcommand{\N}{\mathbb{N}}
\newcommand{\Lc}{\mathcal{L}}
\newcommand{\brac}[1]{\left ( #1  \right )}
\newcommand{\abs}[1]{\left | #1  \right | }
\begin{document}  
\maketitle
\begin{enumerate}
\item We define the differential operator $\Lc(v)(x) := -\partial^2_xv(x) + q(x)v(x)$, then we can reformulate the given problem as
\begin{equation} \label{(1)_reform}
\begin{cases}
\Lc(\tilde u)(x) = f(x), & x \in(0,1) \\
\partial_x \tilde u(0) = \alpha, \\
\tilde u(1) = \beta 
\end{cases} 
\end{equation} 

In order to apply the 'ghost' point technique, we consider (\ref{(1)_reform}) but where the interval on which the problem is defined is slightly extended to $(-h,1)$, i.e.

\begin{equation} \label{(1)_extended}
\begin{cases}
\Lc(\tilde u)(x) = f(x), & x \in(-h,1) \\
\partial_x \tilde u(0) = \alpha, \\
\tilde u(1) = \beta 
\end{cases} 
\end{equation} 

Let $N \in \N$ and we define $h = \frac 1 N$, we then create a mesh for $(-h,1)$, denoted $\Omega_h$, where $\Omega_h = \{x_j\}$, where $x_j := jh, \; j = -1, 0, \dotsc, N$. Let $u_h \in S_h$ and define $u_i = u_h(x_i)$.

We then calculate the second difference of $u_h$ at 0 to be 
 \[D^2_x u_h(0) = \frac{ u_h(-h)-2  u_h(0)+ u_h(h)}{h^2}\] 
\begin{equation} \label{D2_at_0}
= \frac{1}{h^2}(u_{-1}-2  u_0+  u_1)
\end{equation} 

We then rearrange $\alpha = \partial_x \tilde u(0) \approx \frac{1}{2h}(u_1 - u_{-1})$ to get $u_{-1} \approx u_1 -2h\alpha$. Inserting this into (\ref{D2_at_0}) gives us 
\[D^2_x u_h(0) = \frac{2}{h^2}(u_1 - u_0 - h\alpha).\]

We can define the discrete operator $\Lc_h : S_h \to S_h$ given by
\begin{equation} \label{def_of_Lh}
\begin{cases}
\Lc_h(u_h)_0 = -\frac{2}{h^2}(u_1-u_0) +q(x_0)u_0  \\
\Lc_h(u_h)_i = -\frac{u_{i+1}-2u_i+u_{i-1}}{h^2} + q(x_i)u_i, & i = 1, \dotsc, N-1 \\
\Lc_h(u_h)_N = u_N 
\end{cases} 
\end{equation} 

Therefore we can discretise (\ref{(1)_extended}) using $\Lc_h$ by
\begin{equation} \label{discrete_(1)_extended}
\begin{cases}
\Lc_h(u_h)_0 = f(x_0) - 2\alpha/h \\
\Lc_h(u_h)_i = f(x_i), & i = 1, \dotsc, N-1 \\
\Lc_h(u_h)_N = \beta 
\end{cases} 
\end{equation}

\item Using (\ref{def_of_Lh}) and (\ref{discrete_(1)_extended}) we define the matrix
\[A = 
\begin{pmatrix}
\frac{2}{h^2}+q(x_0) & \frac{-2}{h^2} & 0& \ldots  & 0 \\
\frac{-1}{h^2} & \frac{2}{h^2} + q(x_1) & \frac{-1}{h^2} & \ddots & \vdots \\
0 & \ddots & \ddots & \ddots & 0 \\ 
\vdots & \ddots & \frac{-1}{h^2} & \frac{2}{h^2} + q(x_{N-1}) & \frac{-1}{h^2} \\
0 & \dots & 0 & 0 & 1
\end{pmatrix} \in \mathbb{R}^{(N+1) \times (N+1)}, 
\]
and the vectors
\[U = 
\begin{pmatrix}
u_0 \\
u_1 \\
\vdots \\
u_{N-1} \\
u_N
\end{pmatrix}\in \mathbb{R}^{N+1}, \;
B = \begin{pmatrix}
f(x_0) - \frac{2\alpha}{h} \\
f(x_1) \\
\vdots \\
f(x_{N-1}) \\
\beta
\end{pmatrix} 
\in \mathbb{R}^{N+1}.
 \]
So (\ref{def_of_Lh}) and (\ref{discrete_(1)_extended}) can be rewritten as $AU = B$

\item Let $z_h := \Lc_h(v_h)$, recall that $q(x_i) \geq Q^* > 0$, then we calculate that
\begin{equation} \label{lower_bound}
\abs{ \brac{ \frac{2}{h^2} +Q^*}v_i } \leq \brac{\frac{2}{h^2}+ q(x_i)}|v_i|.
\end{equation}   
Recall for $i = 1, \dotsc, N-1$, we have
\[z_i =  -\frac{1}{h^2} v_{i-1}+\brac{\frac{2}{h^2}+q(x_i)}v_i-\frac{1}{h^2}v_{i+1} \] 
which implies, using the triangle rule,
\[\brac{\frac{2}{h^2}+ q(x_i)}|v_i| \leq |z_i| + \frac{1}{h^2}|v_{i-1}| +  \frac{1}{h^2}|v_{i+1}|\]
\begin{equation} \label{up_bound_i_eq_1_to_N}
\leq \frac{2}{h^2}\|v_h\|_{\infty, h} + \|z_h\|_{\infty,h}.
\end{equation}
Similarly, for $i=0$, recall
\[z_0 = \brac{ \frac{2}{h^2} + q(x_0)}v_0 - \frac{2}{h^2}v_1\]
Rearranging and applying the triangle inequality again, gives
\begin{equation} \label{up_bound_i_eq_0}
\brac{\frac{2}{h^2}+ q(x_0)}|v_0| \leq |z_0| + \frac{2}{h^2}|v_{1}| \leq \frac{2}{h^2}\|v_h\|_{\infty, h} + \|z_h\|_{\infty,h}.
\end{equation} 
Applying (\ref{up_bound_i_eq_0}) and (\ref{up_bound_i_eq_1_to_N}) to (\ref{lower_bound}) gives 
\[\brac{ \frac{2}{h^2}+Q^*}\|v_h\|_{\infty,h} \leq \frac{2}{h^2}\|v_h\|_{\infty, h} + \|z_h\|_{\infty,h}, \]
rearranging the above shows we can take $C = 1/Q^*$ when bounding for $i = 0, \dotsc, N-1$.
For $i = N$, we have $z_N = v_N$ so clearly 
\[|v_N| = |z_N| \leq   \|z_h\|_{\infty,h}\]
Therefore the inequality holds for $C = \max\{1, 1/Q^*\}$
\item Since this method depends on the second difference, the order of consistency will be 2 when using the supremum norm. However we require our solution, $\tilde u$ to be in $C^4([-h,1])$
\item Let $b_h = \Lc_h(u_h)$. We know by applying the proof of Theorem 1.7, that for $i = 1, \dotsc, N-1$
\[|\Lc_h(\tilde u)_i - b_i| \leq C_ch^2| \tilde u |_{C^4([-h,1])} \]
for some constant $C_c$. For $i=N$ we have $\Lc_h(\tilde u) = \beta$ and therefore $|\Lc(\tilde u)_N - b_N| = 0$. For $i=0$ we have
\[|\Lc(\tilde u)_0 - b_0| =  \abs{-\frac{2}{h^2}(\tilde u(x_1)- \tilde u(x_0)) +q(x_0)\tilde u(x_0) - f(x_0) + \frac{2\alpha}{h}} \]
\[= \abs{ -D^2_x \tilde u(0)   +q(x_0)\tilde u(x_0) - f(x_0)}\] 
\[ = | -D^2_x \tilde u(0) \underbrace{-\partial^2_x  \tilde u(0)  +q(x_0)\tilde u(x_0) - f(x_0)}_{=0} + \partial^2_x  \tilde u(0)|\]
\[= \abs{\partial^2_x  \tilde u(0) -D^2_x \tilde u(0) } \leq C_c h^2 | \tilde u |_{C^4([-h,1])}, \] 
where in the last step we applied lemma 1.3 in the notes. Thus by applying question three, we have
\[ \| \tilde u - u_h \|_{\infty,h} \leq  C_c \| \Lc_h (\tilde u - u_h)\|_{\infty,h}\]
\[= C_c \| \Lc_h (\tilde u) - \Lc_h(u_h)\|_{\infty,h} \leq C_sC_c h^2| \tilde u |_{C^4([-h,1])} \]
Therefore proving the order of convergence to be two two when using the supremum norm to calculate error. As $h \to 0$, $C_sC_c h^2| \tilde u |_{C^4([0,1])} \to 0$, therefore $u_h \to \tilde u$ as $h \to 0$ in the discrete supremum norm and $u_h$ converges to $\tilde u$.

\item Using this new scheme would mean that the approximation around $x_0$ would not be done using the second difference. It uses a forward difference scheme, which means the new order of convergence for this scheme would be one. However, we would only need the solution to be in $C^3([0,1])$ for this new scheme to work.
\end{enumerate}

\end{document}

